\documentclass{article}

\usepackage[letterpaper]{geometry}
\usepackage{tgpagella}
\usepackage{amsmath}
\usepackage{amssymb}
\usepackage{amsthm}
\usepackage{tikz}
\usepackage{minted}
\usepackage{physics}
\usepackage{siunitx}

\sisetup{detect-all}
\newtheorem{prop}{Proposition}

\title{VIR Results}
\author{Duncan Wilkie et. al.}
\date{1 April 2023}

\begin{document}

\maketitle

The project concerns the Floer homology of 1-1 knots.
Knots of this type can be drawn on a torus, and represented via a 4-tuple $[p, a, b, r]$,
from which the knot can be drawn on the torus (as a quotient of $\mathbb{R}^{2}$) in a Heegaard diagram.
Computing the homology from this diagram is very easy---precisely the reason for the study of this topic.

\section{Drawing}

Code is being written which takes as input the 4-tuple and draws the Heegaard diagram in TikZ.
The correctness of this code depends firstly on the correctness of the underlying tooling,
which we shall not verify,
but also on several semantic propositions about relationships between sides of the diagrams.

Those we may formulate mathematically, and attempt to prove.

The code concerns in large part the relationship between the generators and a coordinate system in TikZ.
Given a Heegaard diagram drawn on a square, this coordinate system is imposed as follows:
the vertical or $y$ axis lies along the left side of the diagram, and the horizontal or $x$ axis
lies along the very bottom of the diagram.
The scale for the axes is such that the top of the square diagram is the line $y = p+1$,
and the point corresponding to the $k$th generator on the left side lies at $y = p + 1 - k$.

The drawing of a single diagram is done via \verb|\diagram{p}{a}{b}{r}|; in the macro body, these arguments become \verb|#1, #2, #3, #4|.
Most of the logic is done using the TikZ-math extension, which permits basic integer and modular integer arithmetic.

There is initial setup of some macro constants when the function is called:
\inputminted[firstline=11, lastline=40]{tex}{../draw/draw.tex}

Then, we define some convenience functions to avoid shortcomings of the DSL:
\inputminted[firstline=41, lastline=57]{tex}{../draw/draw.tex}

Now, we can get into some content.
The following produces the right-hand points corresponding to a left-hand point before applying the \verb|\shift|;
this helps by not needing to differentiate between above and below arcs in the drawing logic.
\inputminted[firstline=58, lastline=69]{tex}{../draw/draw.tex}

Now, for the actual drawing.
This function draws everything from the left side:
\inputminted[firstline=70, lastline=98]{tex}{../draw/draw.tex}

And this one takes care of what that misses:
\inputminted[firstline=99, lastline=135]{tex}{../draw/draw.tex}

With this set up, it's a simple matter of iterating over the points, calling these functions,
and adding details.
\inputminted[firstline=136, lastline=144]{tex}{../draw/draw.tex}

The actual user-facing code embellishes and combines these diagrams to produce high-quality substrates
for contentual mathematical reasoning: the user is expected to use
\inputminted[firstline=148, lastline=155]{text}{../draw/draw.tex}
for standalone diagrams (which merely adds generator numberings) and
\inputminted[firstline=157, lastline=175]{tex}{../draw/draw.tex}
for drawing rectangular subsets of the universal cover diagram.


\begin{prop}
  The following tikzmath function converts a generator number, a side number, and an axis number
  to the corresponding coordinate of the generator in the above-described system.
  The side number is either 0, corresponding to a left-side generator, or 1,
  corresponding to a right-side generator.
  The axis number is either 0, corresponding to the horizontal coordinate, or 1,
  corresponding to the vertical coordinate.
\end{prop}

\begin{prop}
  The following tikzmath function converts a generator number on the left side to
\end{prop}

\begin{prop}
  The
\end{prop}


\end{document}
