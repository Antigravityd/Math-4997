\documentclass{beamer}

\usetheme{metropolis}


\usepackage{tgpagella}
\usepackage{amsmath}
\usepackage{amssymb}
\usepackage{amsthm}
\usepackage{tikz}
\usepackage{minted}
\usepackage{physics}
\usepackage{siunitx}
\usepackage{graphicx}
\usepackage{tikz-cd}
\sisetup{detect-all}
\newtheorem{plm}{Problem}

\title{VIR Progress}
\author{Duncan Wilkie}
\date{3 March 2022}

\usetikzlibrary{math,calc}


\newcommand{\diagram}[4]{ % arguments-to-math key: the four-tuple (p, a, b, r) is represented by (#1, #2, #3, #4)
  \tikzmath{ % everything else
    \points = #1;
    \maxy = \points + 1;
    \maxx = \points;
    \discs = #2;
    \abov = #3;
    \belo = \points - 2*\discs - \abov;
    \shift = #4 - 1;
    \discmidl = \maxy - \discs - 0.5;
    \discmidr = \maxy - mod(\abov + \discs + \shift, \points) - 0.5;
    % annoyingly, these functions don't seem to be able to output coordinates.
    % this is a fix: a function call that emulates a coordinate depending on an axis parameter
    function proj(\vala, \valb, \whaxis) {
      if \whaxis == 0 then {
        return \vala;
      } else {
        return \valb;
      };
    };
    % given an input node number, return the number of the node it's connected to.
    % first coordinate if axis = 0; second otherwise.
    function otherend(\n, \axis) {
      if \n <= \points then { % left side
        if \n <= 2*\discs then { % disc point
          \endpoint = 2 * \discs - \n + 1;
          return proj(abs(\n - \discs - 0.5), \discmidl, \axis);
        } else {
          if \n <= \points - \belo then { % above arc
            \endpoint = mod(\n + \points - 2*\discs + \shift - 1, \points) + 1;
            if \endpoint < \n then { % warps
              return proj(\maxx - \n + 1, 0, \axis);
            } else {
              return proj(\maxx, \maxy - \endpoint, \axis);
            };
          } else { % below arc
            \endpoint = mod(\n + \shift - 1, \points) + 1;
            if \endpoint < \n then { % warps
              return proj(\maxx - \n + 1, 0, \axis);
            } else { % no warp
              return proj(\maxx, \maxy - \endpoint, \axis);
            };
          };
        };
      } else { % right side
        \preimage = mod(\n - \shift - 1, \points) + 1;
        \preerimage = mod(\preimage - \abov - 1, \points) + 1;
        if \preimage <= \abov then { % above arc
          \endpoint = 2*\discs + \preimage;
          if \n - \points < \endpoint then {
            return proj(\maxx - (\n - \points), \maxy, \axis);
          } else {
            return proj(0, \maxy - \endpoint, \axis);
          };
        } else {
          if \preerimage <= 2*\discs then { % disc point
            \endpoint = mod((2*\discs - \preerimage + 1) + \shift + \abov - 1, \points) + \points + 1;
            if \preerimage <= \discs then { % top half
              if \endpoint < \n then { % warps
                return proj(\n - \points - 1, 0, \axis);
              } else { % no warp
                return proj(\maxx - abs(\preerimage - \discs - 0.5), \discmidr, \axis);
              };
            } else { % bottom half
              if \n < \endpoint then { % warps
                return proj(\maxx - (\n - \points), \maxy, \axis);
              } else { % no warp
                return proj(\maxx - abs(\preerimage - \discs - 0.5), \discmidr, \axis);
              };
            };
          } else { % below arc
            \endpoint = mod(\n - \shift - 1, \points) + 1;
            if \n < \endpoint + \points then { % warps
              return proj(\maxx - (\n - \points), \maxy, \axis);
            } else { % no warp
              return proj(0, \maxy - \endpoint, \axis);
            };
          };
        };
      };
    };
    % now actually draw it
    for \i in {1,...,\points}{
      \lx = 0;
      \ly = \maxy - \i;
      \rx = \maxx;
      \ry = \maxy - \i;
      \otherlx = otherend(\i, 0);
      \otherly = otherend(\i, 1);
      \otherrx = otherend(\i + \points, 0);
      \otherry = otherend(\i + \points, 1);
      {
        \node[dot] (c) at (\lx,\ly) {};
        \node[dot] (d) at (\rx,\ry) {};
        \draw (\lx,\ly) -- (\otherlx,\otherly);
        \draw (\rx,\ry) -- (\otherrx,\otherry);
      };
    };
  }
  % bounding square
  \draw (0,0) rectangle (\maxx,\maxy);
}

% Put the diagram in a tikzpicture and draw the node labels
\newcommand{\heegaard}[4]{
  \begin{tikzpicture}[dot/.style = {circle, fill, minimum size=6pt, inner sep=0pt, outer sep=0pt}, dot/.default = 6pt]
    \diagram{#1}{#2}{#3}{#4}
    \draw (0,0) node foreach \i in {1,...,#1} at (-0.5,#1-\i+1) {$\i$};
    \draw (0,0) node foreach \i in {1,...,#1} at (#1+0.5,#1-\i+1) {$\i$};
  \end{tikzpicture}
}

% Given [p, a, b, r] and an x and a y, make an x-by-y grid of Heegaard diagrams.
\newcommand{\ucover}[6]{
  \tikzset{
    tile/.pic={
      \diagram{#1}{#2}{#3}{#4}
    }
  }
  \begin{tikzpicture}[dot/.style = {circle, fill, minimum size=6pt, inner sep=0pt, outer sep=0pt}, dot/.default = 6pt]
    \foreach \x in {1,...,#5}  {
      \foreach \y in {1,...,#6} {
        \tikzmath{
          \xorig = (\x - 1) * #1;
          \yorig = (\y - 1) * (#1 + 1);
        }
        \pic (S) at (\xorig,\yorig) {tile};
      };
    };
  \end{tikzpicture}
}


\begin{document}

\begin{frame}
  \titlepage
\end{frame}

\begin{frame}
  \frametitle{What We're Doing, Again}
    \begin{center}
      \scalebox{0.63}{
        \begin{tikzcd}[ampersand replacement=\&]
          \textrm{Knot Floer homology} \arrow[d] \&  \textrm{of 1-1 knots} \arrow[d] \\
          \textrm{Powerful, but hard-to-compute knot invariant} \arrow[d]
          \&  \textrm{Class of knots one can compute it for} \arrow[d] \\
          \widehat{HFK}(K) = \bigoplus_{i, s}HFK_{i}(K, s) \& \textrm{Can be split into two arcs, each ``simple'' on a torus}
        \end{tikzcd}
      }
    \end{center}
\end{frame}

\begin{frame}
  \frametitle{What We're Doing, Again}
  \begin{itemize}
  \item 1-1 knots are represented by 4 integers $[p, a, b, r]$.
  \item Can read off $\widehat{HFK}$ from diagram constructible from $[p, a, b, r]$
  \end{itemize}
  E.g. $[7, 3, 1, 5] \mapsto$
  \begin{center}
    \scalebox{0.7}{\heegaard{7}{3}{1}{5}}
  \end{center}
\end{frame}

\begin{frame}
  \frametitle{What We've Done}
  \begin{itemize}
  \item Large jumps in grading---why?
    \begin{itemize}
    \item Looked for combinatorial predictors of jumps among $[p, a, b, r]$.
    \item General methods for detecting combinatorial patterns: alien coding, inductive logic programming, symbolic regression,
      syntax-guided synthesis, Berlekamp-Massey algorithm.
    \item Dead end; $[p, a, b, r]$ is unlikely to hold much topological data.
    \end{itemize}
  \item Automatically drawing diagrams
    \begin{itemize}
    \item It's 2023; why use your hands?
    \item \texttt{\\heegaard\{p\}\{a\}\{b\}\{r\},\\ucover\{p\}\{a\}\{b\}\{r\}} $\mapsto$ TikZ diagrams
    \item Works in many cases
    \end{itemize}
  \end{itemize}
\end{frame}

\begin{frame}
  \frametitle{Where We're Going}
  \begin{itemize}
  \item s/grading jumps/multiplicity of top grading/
    \begin{itemize}
    \item In $\bigoplus_{i, s}\widehat{HFK}_{i}(K, s)$, sometimes there is one topmost ``bin,'' and sometimes there are several
    \item What does this say about the knot?
    \item What does this say about the diagram?
    \end{itemize}
  \item Fix drawing bugs
    \begin{itemize}
    \item Sometimes algo is just wrong.
    \item Make warped lines match up in universal cover, so one can see connected regions.
    \end{itemize}
  \end{itemize}

  \qed
\end{frame}
\end{document}
